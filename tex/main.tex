% vim:foldmethod=marker:fmr=<<<,>>> spell nowrap
\documentclass{article}

% <<<
\usepackage[utf8]{inputenc}
\usepackage[total={6.2in, 8in}]{geometry}

\usepackage{amsmath, amsthm, amssymb,amsfonts, mathtools}
\usepackage{graphicx}
\usepackage{float}
\usepackage[percent]{overpic} %writing over pictures
\usepackage{xcolor}
\usepackage{appendix}

\usepackage{mdframed}
\usepackage[colorlinks,urlcolor=blue]{hyperref}
\usepackage{verbatim}
%\usepackage{fancyvrb}
\usepackage{minted}
\renewcommand{\MintedPygmentize}{pygmentize}
\usepackage[ruled,linesnumbered]{algorithm2e}
\setlength{\algomargin}{10pt}
\newcommand{\REM}[1]{\tcp*{\parbox[t]{2.0in}{\raggedright #1}}}
\usepackage[normalem]{ulem}

\usepackage{biblatex}
\addbibresource{main.bib}

\usepackage{listings}
\usepackage{tikz-cd}
\lstset{
basicstyle=\small\ttfamily,
columns=flexible,
breaklines=true
}

\setlength{\parindent}{0pt}

\newcommand{\ket}[1]{\left|{#1}\right\rangle}
\newcommand{\bra}[1]{\left\langle{#1}\right|}
\newcommand{\BS}{\backslash}

\newcommand{\CLR}[2]{\begingroup\color{#1}#2\endgroup}
\newcommand{\R}[1]{\begingroup\color{red}#1\endgroup}
\newcommand{\G}[1]{\begingroup\color{green}#1\endgroup}
\newcommand{\B}[1]{\begingroup\color{blue}#1\endgroup}
\newcommand{\Y}[1]{\begingroup\color{yellow}#1\endgroup}

\newcommand{\vsp}[0]{\vspace*{10pt}\par}
\newcommand{\tcite}[1]{\textit{(\citefield{#1}{year}, \citeauthor{#1})}\:\textit{``\citefield{#1}{title}''}\:\cite{#1}}
\newcommand{\exercise}[1]{\subsubsection*{#1}}
\newcommand{\ans}[0]{\vsp\textbf{Answer: }}

\newcommand{\JProg}{\mathbb{J}}
\newcommand{\AST}{\mathbb{AST}}
\newcommand{\TNET}{\mathbb{TNET}}
\newcommand{\CV}{\mathbb{R}^N}
% Define \set{} command. TODO: Looks too complicated!
\DeclarePairedDelimiterX{\set}[1]{\{}{\}}{\setargs{#1}}
\NewDocumentCommand{\setargs}{>{\SplitArgument{1}{;}}m}
{\setargsaux#1}
\NewDocumentCommand{\setargsaux}{mm}
{\IfNoValueTF{#2}{#1} {#1\,\delimsize|\,\mathopen{}#2}}%{#1\:;\:#2}

% Generic environment for code snippets
\newenvironment{codeverbatim}
  {\VerbatimEnvironment
   \begin{minted}[autogobble,breaklines,fontsize=\footnotesize]{latex}}
  {\end{minted}}
\BeforeBeginEnvironment{codeverbatim}{\begin{mdframed}[nobreak=true,frametitle=\tiny{Source}]}
\AfterEndEnvironment{codeverbatim}{\end{mdframed}}

% LitREPL-compatible environment for code snippets
\newenvironment{python}
  {\VerbatimEnvironment
   \begin{minted}[autogobble,breaklines,fontsize=\footnotesize]{python}}
  {\end{minted}}
\BeforeBeginEnvironment{python}{\begin{mdframed}[nobreak=true,frametitle=\tiny{IPython}]}
\AfterEndEnvironment{python}{\end{mdframed}}

% LitREPL-compatible environment for code results
\newenvironment{result}
  {\VerbatimEnvironment
   \begin{minted}[autogobble,breaklines,fontsize=\footnotesize]{text}}
  {\end{minted}}
\BeforeBeginEnvironment{result}{\begin{mdframed}[nobreak=true,frametitle=\tiny{Result}]}
\AfterEndEnvironment{result}{\end{mdframed}}

% LitREPL-compatible command for inline code results
\newcommand{\linline}[2]{#2}
\newcommand{\st}[1]{\sout{#1}}
\renewcommand{\t}[1]{\texttt{#1}}
% >>>

\setcounter{secnumdepth}{4}
% \RedeclareSectionCommand[runin=false,afterskip=0pt,afterindent=false]{paragraph}

\title{Solutions}
\author{Sergei Mironov}

\begin{document}

\maketitle

\tableofcontents

\section{Introduction}

The book \tcite{Spivak2013CategoryTF}.

\setcounter{section}{2}
\section*{Chapter 2. The Category of Sets}

\setcounter{subsection}{3}
\subsection{Products and coproducts}

\exercise{2.4.1.4}
How many elements does the $\set{a,b,c,d} * \set{1,2,3}$ have?
\ans
12

\exercise{2.4.1.8}
\ans
\vsp (a) No, because $a(b + c) \neq (a+b)c$.
\vsp (b) No, because $x*0 \neq x$.
\vsp (c) Yes.

\exercise{2.4.1.15}

(a) Let $X$ and $Y$ be sets.. construct the "swap map" $s:(X \times Y)\to(Y \times X)$
\ans
$s:(X \times Y)\to(Y \times X) = (,)\circ\langle\pi_2,\pi_1\rangle$

Note: we used angle brackets. Is it really correct?

\vsp
(b) Can you prove that $s$ is a isomorphism using only the universal property for product?

Note: $(f:X \to Y)$ is an isomorphism if $\exists (g:Y \to X): g \circ f = id_X \land f \circ g =
id_Y$

Note: diagram $(f,g,h)$ commutes if $f \circ g = h$
\ans

In universal property of products, put $A$ equal to $Y \times X$ and get $\exists !
g:(Y \times X)\to(X \times Y)$. In a similar way, we have $\exists! s:(X \times Y)\to(Y \times X)$.
We need to show that $g \circ s = id_{(X \times Y)}$ and $s \circ g = id_{(Y\times X)}$.

On the following diagram,

% https://tikzcd.yichuanshen.de/#N4Igdg9gJgpgziAXAbVABwnAlgFyxMJZARgBoAmAXVJADcBDAGwFcYkQBBAAgB0e8AtvC4AhEAF9S6TLnyEUZAAzU6TVu259BwsZOnY8BIuVLEVDFm0QhdUkBgNyii0+bVXOElTCgBzeESgAGYAThACSC4gOBBIZCCM9ABGMIwACjKG8iAhWL4AFjggNBbq1nwwAB5YcDhwAIRBEnah4ZE0MUgmqpbsfGhYAPrkxQnJqRmORta5BUV6IK0RiFGdiADMJe59PAODxM3BYcvxa5s9ZSD9Qwc0iSnpmU4zeYWHi8dxHbGI3aUe12GXnEQA
\begin{tikzcd}
  & A \times B \arrow[ld, "\pi_1"'] \arrow[rd, "\pi_2"]                           &   \\
A &                                                                               & B \\
  & A \times B \arrow[uu, "\exists!id"'] \arrow[ru, "\pi_2"'] \arrow[lu, "\pi_1"] &
\end{tikzcd}

$id = id_{(A \times B)}$ - is the unique identity function. By combining diagrams for $f$ and $g$ we
reduce the $g \circ f$ to the similar case.

\exercise{2.4.2.4}

Would you say that a phone is the coproduct of a \texttt{cellphone} and a \texttt{landline phone}?

\ans
Yes, until we consider other types of phones besides cell- and landline ones.

\exercise{2.4.2.10}

Write the universal property for coproduct in terms of a relationship between..

\ans (TODO: not sure!) $Hom_{Set}(X,A) \sqcup Hom_{Set}(Y,A) \cong Hom_{Set}(X \sqcup Y, A)$

\exercise{2.4.2.13}

TODO

\exercise{2.4.2.14}

TODO


\subsection{Finite limits in Set}

\exercise{2.5.1.2}

\ans $X \times_{Z} Y = \set{(x_1,z_1,y_1), (x_2,z_2,y_2), (x_2,z_2,y_4), (x_3,z_2,y_2),
(x_3,z_2,y_4)}$


\exercise{2.5.1.3}

(a)
\ans Let $X = \set{\Y{1},\R{2},\B{3},\Y{4},\R{5}}; Y = \set{\Y{a},\B{b},\R{c}};$ where
$C = \set{\R{R},\B{B},\Y{Y}}$ \vsp We have:
$X \times_C Y = \set{\Y{1a},\Y{4a},\R{2c},\R{5c},\B{3b}}$

(b)
TODO (obvious).

\exercise{2.5.1.5}

(a) Suppose that $Y = \emptyset$; what can you say about $X \times_Z Y$ ?
\ans $X \times_Z Y = \emptyset$ \vsp

(b) $Z = {1}$; what can you say about $X \times_Z Y$ ?
\ans $\forall X,Y : X \times_Z Y \cong X \times Y$

\exercise{2.5.1.6}

.. Aristotelian space and time ..

$S = R^3;\quad T = R;\quad Y = S \times T;\quad g1 : Y \to S ;\quad g2 : Y \to S$ where $g1,g2$
projects space-time to its components. $X = \set{1};\quad f_1 : X \to S;\quad f_2 : X \to T$ is a
set of one element and its space-time projections.

\vsp
(a) What is the meaning of

\begin{minipage}[t]{0.45\textwidth}
\[
\begin{tikzcd}
W_1 \arrow[r] \arrow[d] & Y \arrow[d,"g_1"] \\
X \arrow[r,"f_2"] & S
\end{tikzcd}
\]
\end{minipage}
\quad
\begin{minipage}[t]{0.45\textwidth}
\[
\begin{tikzcd}
W_2 \arrow[r] \arrow[d] & Y \arrow[d,"g_2"] \\
X \arrow[r,"f_2"] & S
\end{tikzcd}
\]
\end{minipage}

\ans $1$ is associated with its time and space. $W_1$ yields time points of $Y$ corresponding
to $1$'s position. $W_2$ yields the space points corresponding to $1$'s time.

\vsp
(b) Interpret the sets in terms of the center of mass of MIT at the time of its founding.

\ans (TODO: not sure!) Is it just the MIT-relative space and time points?

\exercise{2.5.1.10}

.. Appropriate or misleading olog labels ..

(a) a person whose favorite color is blue - OK
(b) a dog whose owner is a woman - OK
(c) a good fit - Nope. We would say that a good fit requires less or equal width.

\vsp
\vsp
\vsp

\printbibliography

\end{document}

